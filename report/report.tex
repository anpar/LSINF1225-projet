\input{lib.tex}
\doctitle{BarTender : présentation du prototype}

\section{Manuel d'utilisation}
\subsection{Écran d'accueil de l'application}
Lorsque l'application démarre, vous accédez à l'écran
présenté sur la figure \ref{fig:main-menu}. Sur cet écran,
il est possible d'accéder à la carte, de se connecter ou de
s'inscrire.

\begin{figure}[H]
	\centering
	\includegraphics[scale=0.15]{img/main-menu.png}
	\caption{Écran d'accueil de l'application.}
	\label{fig:main-menu}
\end{figure}

\subsection{Connexion et inscription}
Pour se connecter à l'application, il suffit de cliquer
sur le bouton ``Connexion'' de l'écran d'accueil. Vous
arrivez alors sur un écran similaire à la figure \ref{fig:register-login}
sur lequel il vous suffit d'indiquer votre identifiant et votre
mot de passe. Pour s'inscrire sur l'application, il suffit de cliquer
sur le bouton ``Inscription'' de l'écran d'accueil. Vous arrivez alors à un écran similaire à la figure \ref{fig:register-login}. L'inscription
est volontairement simple et ne demande qu'un identifiant, un mot
de passe et sa confirmation. 

\begin{figure}[H]
    \centering
    \begin{subfigure}
				\centering
				\includegraphics[scale=0.15]{img/login.png}
    \end{subfigure}%
    ~ 
    \begin{subfigure}
				\centering
				\includegraphics[scale=0.15]{img/register.png}
		\end{subfigure}
    \caption{Page de connexion à gauche, et d'inscription à droite.}
		\label{fig:register-login}
\end{figure}

\subsection{La carte}
La carte est accessible de plusieurs manières. Premièrement, depuis l'écran d'accueil
présenté à la figure \ref{fig:main-menu}. Ce bouton permet à un client qui ne désire
pas créer un compte de quand même pouvoir consulter la carte. Bien évidemment, la carte
est également accessible une fois connecté. La carte se présente comme sur la
figure \ref{fig:carte}. Il est possible de trier les boissons par noms ou par prix,
en cliquant respectivement sur ``Boisson'' ou ``prix''.

\begin{figure}[H]
	\centering
	\includegraphics[scale=0.15]{img/carte.png}
	\caption{Carte des boissons.}
	\label{fig:carte}
\end{figure}

En cliquant sur une boisson, des détails apparaissent. Ces détails sont bien
sur différents selon que vous soyez connecté en tant que client ou en tant
que serveur. Pour chaque boisson, le client pourra lire une courte
description, un volume, un prix et une note moyenne (donnée par d'autres
clients), comme présenté sur la figure \ref{fig:drink-details}.

\begin{figure}[H]
	\centering
	\includegraphics[scale=0.15]{img/drink-details.png}
	\caption{Détails d'une boisson.}
	\label{fig:drink-details}
\end{figure}

Pour un serveur, ces détails sont un peu différents. Ils contiennent
cette fois des informations sur le stock des boissons : le stock
actuel, le stock maximum, et le seuil minimal (indiquant qu'il est temps
de recommander cette boisson auprès du fournisseur). Ces informations
sont représentés de façon plus visuelle dans une barre d'évolution.

\begin{figure}[H]
    \centering
    \begin{subfigure}
        \centering
        \includegraphics[scale=0.15]{img/stock-full.png}
    \end{subfigure}%
    ~ 
    \begin{subfigure}
        \centering
        \includegraphics[scale=0.15]{img/stock-empty.png}
    \end{subfigure}
    \caption{Détails d'une boisson côté serveur.}
		\label{fig:stocks}
\end{figure}

Comme le montre la figure \ref{fig:stocks}, la couleur de cette barre
change selon que l'on soit en-dessous ou au-dessus du seuil minimal.

\subsection{La recherche avancée}
Le client et le serveur ont tous les deux accès à une fonctionnalité commune,
la recherche avancée, présentée sur la figure \ref{fig:search}. 
Différents critères de recherches sont disponibles : par nom, par catégorie
et sous-catégorie, et en indiquant un prix mimimum et maximum.

\begin{figure}[H]
	\centering
	\includegraphics[scale=0.15]{img/search.png}
	\caption{Recherche avancée.}
	\label{fig:search}
\end{figure}

\input{footer.tex}